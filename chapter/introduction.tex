此处为引言示例,请在introduction.tex中编辑引言。

注意页眉从正文开始。

“啊,连续落了两颗呀!”

“嗯。再有一个就十颗了”

两人兴奋的声音在灭了灯的屋内响彻着。差不多快丑时三刻了——连哭泣的孩子都会安静的时刻,然而灵梦和魔理沙两人却没有安静。

店内被两人占据着,并且所有的灯火都被搞灭了。我既不能读书也不能写日记,便借着从窗户洒漏进来的少量月光移动到了两人那里。

“真拿你们两个没办法,已经足够了吧。像这种‘流星群’也不算什么稀奇物……”

“说什么啊。不是香霖说的嘛?今晚的流星群很厉害,一定会落百颗以上啥的”

“我想确实会落百颗以上……难道你打算都看吗?”

“嗯嗯当然了。都准备好一百多个愿望了呢”

——白天的香霖堂店内。

是关于被迫陪着灵梦和魔理沙两人鉴赏流星的日子的事。

我正在凝视着放在桌子上的新入荷的奇妙物品。虽说是新入荷的不过那物品本身却很陈旧,整体都有些脏。用金属制成的部分有着斑斑锈迹。

这个物品,是由稍大一点的西瓜程度的球和用来支撑它的四只脚构成的。球是用金属做的,只是样子非常奇怪。几个像尺子一般细的金属弯曲连接成圈子组合在一起,就好似用竹子做的手鞠,是个空隙很多的球体。并且那些金属圈,还分为能够单个自由回转的和被固定在脚上不能动的两种。

遗憾的是有几个金属圈由于生锈的原因,不能流畅地回转。这个样子的话就不成商品了,所以我在想能否通过自己的手将它再生。

“这满是缝隙的奇怪的地球仪是什么?”

“这不是地球仪哦,魔理沙。还有,你何时进到店里来的?”

“我还以为地球开了一个洞呢”

魔理沙问我,如果不是地球仪的话那究竟是什么?

所谓地球仪,正如其名是地球的模型。幻想乡的住民对自己所居住的星球了解甚少。这是因为幻想乡存在于占据着地球上很少一部分的日本的,极小一部分的深山里,而且还不能从里面走出去。

不过,并不是说外面的情报以及道具就流不进来。地球仪也是从外面世界流入的一个道具,通过它我们就可以了解我们所居住的地球。虽然从知识面来讲已经知道些很细微的事了,但对于幻想乡的人来讲,还不能把自己所居住的大地和知识上的地球联系起来。所以就算说地球开了一个洞,也会很容易就相信了。

然而,看起来像地球仪的这个道具决不是地球仪。是用来测量和地球一样的,平时离幻想乡很近,却又不被详细了解的某种东西的道具。

“这个是名叫‘浑天仪’的道具。地球仪是作为认识地球的道具的话,浑天仪就是用来认识宇宙的道具”

浑天仪,既是一种非常复杂的道具,却又仅仅是测量星星位置的东西而已。

不过它之所以复杂是有原因的。星星看起来好似只是浮着而已,但想要准确测定其位置是很难的。既不能用尺子去量,而且就像远处的地面有山啊森林什么的一样高度也不同。如果夜空也像坐标纸那样引有线,或是有很多作为基准的不动的星星的话就简单了,可那当然是不可能的。观测在双手无法触及的地方的,又没有可以作为基准用的东西的星星的位置这件事,自古以来为难了很多的天文学者。为了解决这个问题,浑天仪便不得不成为一个复杂的道具。